\begin{abstract}
    There is a rapid growth in the number of available biological datasets due
    to the advent of high-throughput data collection instruments combined with
    cheap compute infrastructure.  Datasets from modern instruments
    enable analysis of biological data at different levels, from small DNA
    sequences through larger cell structures, and up to the function of entire
    organs. These new datasets have brought the development of new software
    tools and packages for analysis. This leads to the potential for novel
    insights to the underlying biological mechanisms regarding the development
    and progression of diseases such as cancer. 
    
    The heterogeneity of these biological datasets require researchers to tailor
    the exploration and analyses using a range of different tools and systems.
    However, despite the need for using a range of tools, few of these provide
    standard interfaces for analyses implemented using different programming
    languages and frameworks. In addition, because of the many tools, input
    parameters, and reference to databases, it is difficult to report the
    details of an analysis correctly. The lack of such details complicates the
    reproducing of original results and reuse of the analyses on new datasets.
    This increases both time for analysis  and leaves unrealized potential for
    scientific insights.

    This dissertation argues that we can develop unified systems for
    reproducible exploration and analysis of high-throughput biological datasets.
    We propose an approach, \gls{sme}, that orchestrates the
    execution of analysis pipelines and data exploration applications. We
    realize \glspl{sme} using software container technologies together with
    well-defined interfaces, configuration, and orchestration.  It simplifies
    the development of such applications, and provides detailed
    information to reproduce the analyses.
    
    Through this approach we have developed different applications for analyzing
    high-throughput \gls{dna} sequencing datasets, and exploring gene expression
    data integrated with questionnaires, registry, and online databases. The
    evaluation shows how we effectively capture provenance in analysis pipelines
    and exploration applications. This simplifies reproducing and sharing of
    methods and tools. 
    
\end{abstract}

