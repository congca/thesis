\begin{abstract}
    There is a rapid growth in the number of available biological datasets due
    to the decreasing cost of data collection. This brings opportunities to gain
    novel insights to the underlying biological mechanisms in the development
    and progression of diseases such as cancer.  The wide range of different
    biological datasets has led to the development of a wealth of software
    packages and systems to explore and analyze these datasets.  However, there
    are few tools that are designed with the full analysis pipeline in mind,
    from raw data into interpretable results. While the tools are used to
    provide novel insights in diseases, there is little emphasis on reporting
    and sharing information about tool versions, input parameters, and other
    information that can help others use the same known methods on their own
    datasets.  This leads to unnecessary difficulties to reuse known methods,
    and difficulties in reproducing analyses, increasing the analysis time and
    leaves unrealized potential for scientific insights.

    This dissertation argues that, instead, we can design a unified approach
    that integrates disparate systems and data into fully reproducible
    biological data analysis frameworks. In particular, we show how software
    container technologies together with well-defined interfaces,
    configurations, and orchestration provide the necessary foundation to build
    reproducible analysis pipelines for biological datasets, as well as highly
    interactive data exploration applications.

    We show the need and feasibility of our approach through a number of
    different applications for analyzing and exploring biological datasets.
    These applications were developed to meet the requirements of researchers in
    systems epidemiology and precision medicine. We evaluate the approach
    through these systems using real datasets. Our results show that our
    approach can be used to enable reproducible data analysis and exploration of
    high-throughput biological datasets while still providing the performance of
    related systems.
    
\end{abstract}

