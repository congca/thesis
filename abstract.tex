\begin{abstract}
    There is a rapid growth in the number of available biological datasets due
    to the advent of high-throughput data collection instruments combined with
    cheap compute infrastructure.  
    Modern instruments enable the analysis of biological data at different
    levels, from small DNA sequences through larger cell structures, and up to
    the function of entire organs. 
    These new datasets have brought the need to develop new software
    tools and packages to enable novel insights into the underlying biological
    mechanisms in the development and progression of diseases such as cancer. 
    
    The heterogeneity of biological datasets require researchers to tailor
    the exploration and analyses with a range of different tools and systems.
    However, despite the need for their integration, few of them provide
    standard interfaces for analyses implemented using different programming
    languages and frameworks. In addition, because of the many tools, different
    input parameters, and references to databases, it is necessary to record
    these correctly. The lack of such details complicates the reproducing of
    original results and the reuse of the analyses on new datasets. This
    increases the analysis time and leaves unrealized potential for scientific
    insights.
 
    This dissertation argues that we can develop unified systems for
    reproducible exploration and analysis of high-throughput biological
    datasets.  We propose an approach, \Glsentryfirstplural{sme}, for developing
    data analysis pipelines and data exploration applications in cancer
    research.  We realize \glspl{sme} using software container technologies
    together with well-defined interfaces, configuration, and orchestration. It
    simplifies developing such applications, and provides detailed
    information needed to reproduce the analyses.
    
    Through this approach we have developed different applications for analyzing
    high-throughput \gls{dna} sequencing datasets, and for exploring gene
    expression data integrated with questionnaires, registry, and online
    databases. The evaluation shows how we effectively capture provenance in
    analysis pipelines and data exploration applications. It simplifies
    reproducibility through the sharing of methods, tools, and data, and is
    fundamental to science. 
    
\end{abstract}

