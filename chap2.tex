From the discovery of the \gls{dna} structure by Watson and Crick in
1953\cite{watson1953molecular} to the sequencing of the human genome in 2001
\cite{venter2001sequence,international2001initial} and the massively parallel
sequencing platforms in the later years[6], the scientific advances have been
tremendous. Today, single week-long sequencing runs can produce as much data as
did entire genome centers just years ago.\cite{kahn2011future}  These
technologies allow researchers to collect data faster, cheaper and more
efficient, now making it possible to collect the entire genome from a patient in
less than a days work.

In this chapter we give a background in the different aspects of analyzing and
exploring biological datasets. We highlight the necessary processing steps from
data generation and to interpretation of results. 

\section{High-Throughput Datasets in Research and Medicine} 
\gls{dna} sequencing is the process of determining the order of nucleotides
within a strand of \gls{dna}. 

Precision medicine uses patient-specific molecular information to diagnose and
categorize disease to tailor treatment to improve health
outcome.\cite{national2011toward} Important research goal in precision medicine
are to learn about the variability of the molecular characteristics of
individual tumors, their relationship to outcome, and to improve diagnosis and
therapy.\cite{tannock2016limits} International cancer institutions are therefore
offering dedicated personalized medicine programs, but while the data collection
and analysis technology is emerging, there are still unsolved problems to enable
reproducible analyses in clinical settings. For cancer, high throughput
sequencing is the main technology to facilitate personalized diagnosis and
treatment since it enables collecting high quality genomic data from patients
at a low cost. 

\subsection{Sequencing} 
TODO: What is sequencing data, how is it obtained and what does it look like. 
\subsection{Gene expression} 
TODO: What is gene expression, how is it obtained, and what does it look like.

\section{Norwegian Women and Cancer} 
The \gls{nowac} study is a prospective population-based cohort that tracks 34\%
of all Norwegian women born between 1943–57.\cite{lund2007cohort} We started the
data collection in \gls{nowac} in 1991 with questionnaires to cover, among
others, the use of oral contraceptives and hormonal replacement therapy,
reproductive history, smoking, physical activity, breast cancer, and breast
cancer in the family.  We also integrate the Norwegian Cancer Registry, the
register of the National Mammographic Screening Program, and the register of
death certificates in Statistics Norway.  In addition to the questionnaire data,
the \gls{nowac} biobank now contain blood samples from 50 000 women, as well as
tumor tissue samples. From the biological samples we have generated gene
expression, miRNA, methylation, and RNA sequencing datasets. 

From the \gls{nowac} cohort we have have published a number of research papers
that investigate the questionnaire data\cite{find-some-papers}. We have also
used the gene expression datasets to explore ... and interactions between the
tumor and the blood systemic response of breast cancer
patients.\cite{dumeaux2017interactions} While we have studied interesting
patterns and questions, there are still unexplored areas in the available
datasets.

\subsection*{Traditional Data Analysis} 
In \gls{nowac} we have stored the questionnaire data on a in-house database
backed up to an independent storage node. Traditionally, researchers had to
apply to get data exported from the database by an engineer. This was typically
done through SAS scripts that did some preprocessing, e.g. selecting
applicable variables or samples, before the data was shared
to researchers as SAS data files. The downstream analysis was typically done in
SAS as well on local computers. Traditionally researchers have used e-mail to
communicate and share data analysis scripts, and there has not been a central
hub to share scripts or data. To store the biological data we have used a
dedicated node on the local super computer cluster. Because of the many packages
in Bioconductor\footnote{bioconductor.org} we have used R to analyse the
different gene expression datasets on the projcet. However like with SAS the
researchers have typically shared scripts through e-mail and there has not been
a tradition to version control the analysis scripts. 

To enable more researchers to benefit from the unique datasets in \gls{nowac} we
wanted to improve the way researchers access data as well as how they share
their analyses. 

Specifically in the \gls{nowac} project we have developed i) a software package
in the R programming language that contain all datasets and useful utility
functions to work with the data, and ii) a graphical user interface for
non-programmers that wish to access the available datasets, and iii) best
practices for researchers that want to work with \gls{nowac} datasets. In the
next sections we describe the software packages and best practices. 

\subsection{Data management} 

\subsection{Data processing} 

\subsection{Best Practices} 



\section{Preprocessing} 

\section{Analysis Pipelines}
% Watchdog:
% https://bmcbioinformatics.biomedcentral.com/articles/10.1186/s12859-018-2107-4
% maybe this one
% http://gopherdata.io/post/more_go_based_workflow_tools_in_bioinformatics/

\section{Interactive Applications} 
