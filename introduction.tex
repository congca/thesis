% observation
There is a rapid growth in the number of available biological datasets due to
the decreaseing cost of data collection. This brings opportunities to gain
novel insights to the underlying biological mechanisms in the development and
progression of diseases such as cancer.  
% LAB: ...that can be used to develop novel diagnostics tests and drugs for
% treatment?
The wide range of different biological
datasets has led to the development of a wealth of software packages and systems
to explore and analyze these datasets.  
% LAB: these tools, er det software packages eller systems?
% LAB: Jeg er egentlig ikke enig med dettte (target only a small part) eller så
% er det rart skrevet. Det finnes mange komplette pipelines for mange
% forskjellige analyser. 
% LAB: Er det viktig å ha dette med her? Virker mer en challenge.
However, these tools typically target
only small portions of the deep analysis pipeline required to transform the raw
data into interpretable results. While the tools are used to provide novel
insights in diseases, 
% LAB: jeg ville her ha sagt det som kommer i siste setning. Dvs det brukes
% altfor mye tid på å sette opp analysene og tolke resultatene
there is little emphasis on reporting and sharing
information about tool versions, input parameters, and other information that
can help others use the same methods on their own datasets. 
% LAB: resuse known methods; eller develop new pipelines (by not reusing
% previous pipelines)?
This leads to
unneccesary difficilties to reuse known methods, and difficulties in reproducing
analyses, which leads to 
% LAB: det, men også at det tar mye lenger tid å komme frem til ett resultat
unrealized potential for scientific insights and
% LAB: hva er commercial use?
commercial use in the available datasets. 

% challenge
% LAB: spesifisere at det er "computer science" challenges. Vi nevner jo ikke
% alle disse biologiske og statistikk tingene (som også kan være litt
% challenging :)
There are several challenges for researchers to analyze and explore biological
datasets. These challenges are common for large datasets such as high-throughput
sequencing data that require long-running, deep analysis pipelines, as well as
smaller datasets, such as microarray data, that require complex, but
short-running analysis pipelines.  The first is the time and knowledge required
to find and set up the necessary analysis tools to start analyzing a modern
biological dataset.  The second is ensuring the correct input parameters, tool
versions, database versions, and dataset versions when analyzing, and reporting
analysis results to enable reproducible science. The final challenge is the need
for flexible and highly configurable analysis tools to support the large
variation in study design and data types.
% LAB: blir "data exploration" litt borte i challenges? (og observations)? Er
% ikke det en viktig challenge å kombinere denne med en pipeline? Og bare å
% implementere en app som har de rette analysene, visualiseringene, og
% databasene?

% related 
% LAB: kan dette løftes til nivået over enkelt verktøy? Galaxy er en
% bioinformatics workflow manager. Pachyderm er en container manager med
% provenance...
% LAB: De tre hoved områdene jeg kommer på i farten er: bioinformatics workflow
% mangers (Galaxy, CWL/toil, Bioconductor( R), big data anlaytics (Spark,
% Pachyderm), data exploration (webapps, R, visualization tools).
As a result, there are a wealth of approaches and systems to enable analysis of
the complex biological data. To develop deep analysis pipelines,
Galaxy\cite{galaxy} has for a long time provided a simple interface to set up
and execute analysis pipelines for genomic datasets. 
% LAB: litt uenig. Mange vil si at Galaxy er veldig flexibelt, så lengde du kan
% bruke Galaxy tools. Men er mye arbeid å lage en ny tool. Det er også ett stort
% system som er krevende å veldikeholde. Litt usikker på hvordan visualisering
% er for tiden der.
However, the Galaxy system
is less effective for explorative and flexible
analyses.\cite{spjuth2015experiences} 
% LAB: hva med data exploration?
% LAB: kunne Spark også tas her? Dvs data analytics systemer, som ikke er
% tilpasset bioinformatics. Men nyttige som byggesteiner.
Pachyderm is a system for
developing more flexible analyses that support comparison of pipeline runs from
different workflow configurations and datasets.\cite{pachyderm} However it has
yet to see wide-spread adpotion in Bioinformatics.  
% Blir litt vel spesifikt her. Hva er det overordnede poenget? Hva er felles for
% disse?
New initiatives such as the
\gls{cwl} provide users a standardized way of describing an analysis pipeline
and has multiple implementations such as the reference implementation
cwl\_runner,\footnote{\url{github.com/common-workflow-language/cwltool}}
Arvados,\cite{arvados} Rabix,\cite{rabix} Toil,\cite{toil} Galaxy,\cite{galaxy}
and AWE.\cite{awe} While these systems provide system for batch-processing of
large datasets, systems such as Shiny and OpenCPU provide interactive interfaces
to the R programming language and the many packages for biological data
processing in Bioconductor.  
% LAB: En observation? Eller konklusjon om at alle "hoved kategoriene" har
% mangle?
With the addition of new datasets and methods every
year, it seems that analysis of biological data requires a wide array of
different tools and systems.

% our solution
This dissertation argues that, instead, we can design a unified approach that
integrates disperate 
% LAB: tools eller systems? Er stor forskjell på hva folk legger i disse. Jeg
% vil si at approach bør dekke systems for å kunne bruke feks Spark, R, og
% webapps.
tools and data into fully reproducible biological data
analysis frameworks.  
% LAB: er det prinsipielle at det er en container? Eller noe annet?
In particular, we show how software container
technologies, 
% LAB: Docker blir for implementation spesifikt
such as Docker, provide the necessary foundation to build
reproducible environments for any analysis pipeline, as well as a suitable
environment to package an entire data expliration application. 

The resulting approach as several key advantages when implementing systems to
analyze and explore biological data:
\begin{itemize} 
    % LAB:  challenges var:
    % 1. time and knowledge required set up the necessary analysis tools
    % 2. ensuring the correct input parameters, tools ...
    % 3. need for flexible and highly configurable analysis tools 
    
    % jeg ser ikke hvilken challenge denne løser, er det #3?
    \item It supports applications using tools written in any programming
        language, using open standards to communicate between tools and systems,
        enabling new systems to combine the wealth of existing software. 
    % egentlig heller ikke hvilken denne løser, kanskje halve "time and
        % knowledge"
    \item It simplifies the sharing of components and tools, minimizing the time
        to start analyzing and exploring data. This allieviates the tedious
        task of installing tools and their dependencies. 
    % dette er challenge 2
    \item It enables reproducible research by packaging applications
        and tools within self-contained environments.
    % dette var ikke nevnt som en challenge
    \item It does not add any unnecessary storage or computational  overhead to
        current approaches
    % knowledge biten av #1 ser ikke ut til å være løst
    % "unified approach" nevnes ikke
\end{itemize} 

% LAB: trenger ikke å nevne Docker
% LAB: ett poeng som blir borte her er at det ikke bare er laget prototyper, men
% at det er gjort motsatt; det er først laget ting våre samarbeidspartnere har
% behov for; deretter er mer generelle deisgn prinsipper trekket ut
We implement our approach thorugh a series of applications and tools built on
top of a stack of open source systems including Docker as the common foundation.
We evaluate the approach through these systems using real datasets and show its
viability. 

From a longer-term perspective we discuss the general patterns for implementing
modern data analysis systems for use in precision medicine and dicuss why our
approach is a suitable option. As more datasets are produced every year,
research will depend on systems being easy to pick up, and provide the necessary
functionality to reproduce and share the analysis pipelines. 

\emph{Thesis statement}:
A common development model based on software container infrastructure can
efficiently provide reproducible and easy to use environments to develop
applications for exploring and analyzing biological datasets. 

\section{Problems with Data Analysis and Exploration in Bioinformatics}

% LAB: Er kanskje bedre "å snu" disse setningene. Dvs disse systemene/tools har allerede løst mange problemer, men har fortsatt limitations. Kan egentlig kopiere oppbygningen fra Zaharia
Shell scripts have traditionally been the de facto standard for building
bioinformatic analysis pipelines. Luckily there is a move towards using more
sophisticated approaches with workflow and pipeline mangers such as
Galaxy\cite{galaxy} and the \gls{cwl}\cite{cwl}, that simplify setting up the
pipeline, maintaining, and updating it. For exploring biological data there are
a range of tools, such as Cytoscape\cite{cytoscape} and Circos\cite{circos},
% more tools? other tools? 
that support importing an already-analyzed dataset to visualize and browse the
data. 

Although there are efforts to develop tools to help researchers explore and
analyze biological datasets, they current tools have several drawbacks:

% LAB: (Tenker høyt) Slik ville jeg beksrevet limitations: 
% data exploration tools er ikke relevante
% decoupling...
% not reproducible
% too complex: (i) hard to maintain, (ii) hard to scale/ optimize, (iii) hard to develop
% Siste burde splittes opp, men det fant jeg ikke ut hvordan.

\begin{enumerate}
% LAB: Hvis det skal være ett ord for limitation kan Complexity fungere
% Analysene starter som enkle script men de utvides med mer og mer funksjonalitet. 
% Diffucult to maintain codebase. Does not utilize benefits of modern infrastructure systems.
% Blir mer kostbar, dårligere kvalitet, og ikke skalerbar
    \item \textbf{Complex infrastructure and maintenance:} Pipeline managers
        often require complex compute infrastructure and are difficult to keep
        up to date. 
% LAB: Denne og reproducibility beskriver vel samme limitation (Reproducility)
% og teksten bør si: limitation; description/ example of limitation; consequences
    \item \textbf{Data provenance:}  There is little support tracking changes
        of a biological dataset or output datasets, making it difficult to
        compare analysis output. 
    \item \textbf{Reproducibility:} While there are tools for analyzing most
        data types today, there is little or no effort to fully document the
        entire pipeline from raw data to interpretable results. This includes
        tool versions, parameters, data, and databases. 
% LAB: Ett ord (Reusability)
% Her tenker jeg at hovedproblemet er at verktøyene ikke er relevante for biologene som gjør interpratation...fordi at...så de ender ofte opp med å enten manuelt gjøre ting eller programmere sine egne messy ting
    \item \textbf{Limited extensibility:} Data exploration tools are often
        developed as a single unity, making it difficult to reuse parts of the
        application.
    \item \textbf{Decoupling:} Data exploration tools are often decoupled from
        the statistical analyses, making it a difficult exercise to document and
        retrace the analyses behind the results. 
\end{enumerate} 

% vet ikke helt as... 
Because of these drawbacks, an approach for reproducible data analysis and
exploration would have significant benefits in usability and 
% LAB: transparency er nytt
transparency of
complex analyses of biological data.

\section{The X Model/approach/etc.} 
    The Solution (?). Talk about the ideas behind building apps on top of
    containers.  
% LAB: & how limiations are solved

\section{Systems Implemented with the X Model/approach/etc.} 
% LAB: walrus, Kvik, etc syens jeg hører til i "The solution". Her ville jeg heller beskrvet applications. Dvs hva er laget? hva slags problem løser den? hva er det brukt til?
% Dvs:
% 1. NOWAC package (men merk at data managment challenges/limitations ikke er nevnt før). Brukt til å få kontroll over NOWAC data
% 2. Pippelinen. Standardi for NOWAC analyser.
% 3. n=1, RNA-seq, ... og andre per-prosjekt-pipelines. Brukt for diverse prosjekter.
% 4. MIxT
% 5. Kvik Pathways, ... og andre små visualiseringer
% 6. Bidrag til forskjellige andre paper (den nature methods saken, etc)
% ...Combined these demonstrate how the X approach has been used for a full....

We have used the \gls{x} to implement both batch processing systems targeted at
high-throughput analysis pipelines, as well as interactive data
exploration systems for interactively exploring the results and emerging
patterns from these analyses.  We discuss the different systems and areaswe have
implemented. 

\textbf{Deep analysis pipelines}. 
Analysis of high-throughput sequencing datasets requires deep analysis pipelines
with a large number of steps that transform raw data into interpretable
results\cite{diao2015building}. We used our approach to implement walrus, a tool
that lets users create and run analysis pipelines. In addition, it tracks full
provenance of the input, intermediate, and output data, as well as tool
parameters. With \emph{walrus} we have successfully built analysis pipelines to
detect somatic mutations in breast cancer patients. 

\textbf{Interactive exploration}. Analysis pipelines and workflows typically
require researchers to browse and explore the final output. In addition it may
be useful to further explore results by modifying analysis parameters to execute
new analyses.  To develop data exploration applications that interface with the
underlying statistical analyses we have built \emph{Kvik}. Kvik allows
applications written in any modern programming language to interface with the
wealth of bioinformatics packages in the R programming language, as well as
information available through online databases. We have used Kvik to develop the
\gls{mixt} system for exploring and comparing transcriptional profiles from
blood and tumor samples in breast cancer patients. 


\section{Summary of Results} 



\section{List of papers} 
This section contains a list of papers along with short descriptions and my
personal contribution to each paper. 
\capstartfalse
\begin{table}[H]
    \centering
    \begin{tabular}{ | l | p{9.5cm} | }
    \hline
         Title & Kvik: three-tier data exploration tools for flexible analysis
         of genomic data in epidemiological studies \\ \hline
         
         Authors & \textbf{Bjørn Fjukstad}, Karina Standahl Olsen, Mie Jareid,
         Eiliv Lund, and Lars Ailo Bongo \\ \hline
         
         Description & The initial description of Kvik, and how we used it to
         implement Kvik Pathways, a web application for browsing biologicap
         pathway maps integrated with gene expression data from the \gls{nowac}
         cohort. 
         \\ \hline
         
         Contribution & Designed, implemented, and deployed Kvik and Kvik
         Pathways. Evaluated the system and wrote the manuscript. \\ \hline
         
         Publication date & 15 March 2015 \\ \hline 

         Publication venue & F1000 \\ \hline
         
         Citation & \cite{fjukstad2015kvik} \bibentry{fjukstad2015kvik} \\
         \hline 
    \end{tabular}
    \label{p1}

\end{table}
% \hfill 
\begin{table}[H]

    \begin{tabular}{ | l | p{9.5cm} | }
    \hline
         Title & Building Applications For Interactive Data Exploration In
         Systems Biology. \\ \hline
         
         Authors & \textbf{Bjørn Fjukstad}, Vanessa Dumeaux, Karina
         Standahl Olsen, Michael Hallett, Eiliv Lund, and Lars Ailo Bongo.  \\
         \hline
         
         Description & Describes how we further developed the ideas from Paper 1
         into an approach that we used to build the \gls{mixt} web application. 
         \\ \hline
         
         Contribution & 
         Designed, implemented, and deployed Kvik and the \gls{mixt} web
         application.  Evaluated the system and wrote the manuscript. 
         \\ \hline
         
         Publication date & 20 August 2017. \\ \hline  

         Publication venue & The 8th ACM Conference on Bioinformatics,
         Computational Biology, and Health Informatics (ACM BCB) August 20–23,
         2017.  \\
         \hline
         
         Citation & \cite{fjukstad2017building} \bibentry{fjukstad2017building}
         \\ \hline 
    \end{tabular}
    \label{p2}
    
\end{table}
% \hfill 
\begin{table}[H]
    
    \centering
    \begin{tabular}{ | l | p{9.5cm} | }
    \hline
         Title & Interactions Between the Tumor and the Blood Systemic Response
         of Breast Cancer Patients \\ \hline
         
         Authors & Vanessa Dumeaux, \textbf{Bjørn Fjukstad}, Hans E Fjosne,
         Jan-Ole Frantzen, Marit Muri Holmen, Enno Rodegerdts, Ellen
         Schlichting, Anne-Lise Børresen-Dale, Lars Ailo Bongo, Eiliv Lund,
         Michael Hallett.  \\ \hline
         
         Description & Describes the \gls{mixt} system which enables
         identification of genes and pathways in the primary tumor that are tightly
         linked to genes and pathways in the patient \gls{sr}. 
         \\ \hline
         
         Contribution & 
         Designed, implemented, and deployed the \gls{mixt} web application.
        Contributed to write the manuscript. 
         \\ \hline
         
         Publication date & 28 September 2017. \\ \hline  

         Publication venue &  PLoS Computational Biology \\ \hline
         
         Citation & \cite{dumeaux2017interactions}
         \bibentry{dumeaux2017interactions}
         \\ \hline 
    \end{tabular}
    \label{p3}
    
    \hfill 

    \begin{tabular}{ | l | p{9.5cm} | }
    \hline
         Title & A Review of Scalable Bioinformatics Pipelines \\ \hline
         
         Authors & \textbf{Bjørn Fjukstad}, Lars Ailo Bongo. \\ \hline
         
         Description & This review survey several scalable bioinformatics
         pipelines and compare their design and their use of underlying
         frameworks and infrastructures.      \\ \hline
         
         Contribution & 
         Wrote the manuscript.  \\ \hline
         
         Publication date & 23 October 2017 \\ \hline  

         Publication venue & Data Science and Engineering 2017. \\ \hline
         
         Citation & \cite{fjukstad2017review} \bibentry{fjukstad2017review} \\
         \hline 
    \end{tabular}
    \label{p4}
\end{table}
\begin{table}[H]
    \centering
    \begin{tabular}{ | l | p{9.5cm} | }
    \hline
         Title & nsroot: Minimalist Process Isolation Tool Implemented With
         Linux Namespaces.  \\ \hline
         
         Authors & Inge Alexander Raknes, \textbf{Bjørn Fjukstad}, Lars Ailo Bongo. \\ \hline
         
         Description & Describes a tool for process isolation built using Linux
         namespaces.          \\ \hline
         
         Contribution & Contributed to the
         manuscript, specifically to the literature review and related works.
         \\ \hline
         
         Publication date & 26 November 2017 \\ \hline  

         Publication venue & Norsk Informatikkonferanse 2017. \\ \hline
         
         Citation & \cite{fjukstad2017review} \bibentry{fjukstad2017review} \\
         \hline 
    \end{tabular}
    \label{p5}
\end{table}
\begin{table}[H]
    \centering
    \begin{tabular}{ | l | p{9.5cm} | }
    \hline
         Title & Transcription factor PAX6 as a novel prognostic factor and
         putative tumour suppressor in non-small cell lung cancer \\ \hline
         
         Authors & Yury Kiselev, Sigve Andersen, Charles Johannessen, 
         \textbf{Bjørn Fjukstad}, Karina Standahl Olsen, Helge Stenvold, Samer
         Al-Saad, Tom Dønnem, Elin Richardsen, Roy M Bremnes, and Lill-Tove
         Rasmussen Busund.\\ \hline
         
         Description & This paper explores the possibility of using the PAX6
         transcription factor as a prognostic marker in non-small cell lung
         cancer. 
         \\ \hline
         
         Contribution & Did the analyses to explore association between PAX6
         gene expression and PAX6 target genes. 
         \\ \hline
         
         Publication date & 22 March 2018 \\ \hline  

         Publication venue & Scientific Reports 2018. \\ \hline
         
         Citation & \cite{kiselev2018transcription} \bibentry{kiselev2018transcription} \\
         \hline 
    \end{tabular}
    \label{p6}
\end{table}


\begin{table}[H]

    \centering
    \begin{tabular}{ | l | p{9.5cm} | }
    \hline
         Title & Reproducible Data Analysis Pipelines in Precision Medicine \\
         \hline
         
         Authors &  \textbf{Bjørn Fjukstad}, Vanessa Dumeaux, Michael Hallett,
         Lars Ailo Bongo\\ \hline
         
         Description & This paper outlines how we used the container centric
         development model to build walrus. 
         \\ \hline
         
         Contribution & Design, implementation and evaluation of walrus. Wrote
         the manuscript. 
         \\ \hline
         
         Publication date & TBA \\ \hline  

         Publication venue & TBA \\ \hline
         
         Citation & \cite{walrus} \bibentry{walrus} \\
         \hline 
    \end{tabular}
    \label{p6}
\end{table}

% Husk historien: Fra rå data gjennom komplekse analyse pipeliner og helt til
% forskere kan svømme rundt i resultatene. 

\section{Dissertation Plan} 
This thesis is organized as follows. Chapter 2 describes the characteristics of
state-of-the-art biological datasets, the analysis required to extract knowledge
from these, and the available tools and analysis frameworks. Chapter 3 describes
in detail how we use a container centric development model to build a tool,
walrus, to develop and execute deep analysis pipelines. In Chapter 4 we describe
how we used the same model to develop applications to interactively explore
results from statistical analyses.  Finally, Chapter 5 concludes the work and
discusses future directions. 

