
\chapter{Introduction}

% Thesis project two parts: 
% 1: Systems for exploring genomic data. Advanced statistical analyses from the
% web-browser or any other platform. 
% 2:  Analysis pipelines for large genomic datasets. Provenance of data. 
% 
% Common: 
% - Containers
% - Reproducibility
%     - 1: An application is an R package with the analyses. Do not need to
%     pre-compute results, can call functions etc. directly from the app. Build an
%     app as containers that can be shared. 
%     - 2: An analysis is a list of tools with 
%     input/output data. We package each tool in a container (snapshot-ish) 

\emph{Observation}:
Collecting large biological datasets have never been simpler
or cheaper. Data generation is growing at an unprecedented rate. There are a
wealth of software packages and systems to analyze these datasets, but they
often target only a small portion of the full analysis pipeline from raw data to
interpretable results.

\emph{Challenges}:
\begin{enumerate*}[label=(\roman*)]
    \item Giving a fully transparent overview of the data transformations from
        raw data to interpretable results; 
    \item Sharing an analysis pipeline across software platforms and research
        groups; 
    \item Providing full provenance of data, tools, and tool parameters; 
    \item Develop end-user applications that interface with the advanced
        statistical analyses required to explore biological datasets.  
    
\end{enumerate*} 

\emph{Related work}:
There are a wealth of related work and systems. To build analysis pipelines we
have systems such as
\begin{enumerate*}[label=(\roman*)]
    \item CWL and its implementations: cwl\_runner, Arvados, Rabix, Toil, Galaxy,
        AWE; and 
    \item Pahchyderm. 
\end{enumerate*}
When the data is analyzed and ready for further exploration, we have systems
such as
\begin{enumerate*}[label=(\roman*)]
    \item OpenCPU;
    \item RStudio and Shiny; 
    \item Renjin an R interpreter built on top of the JVM. 
\end{enumerate*}
 

\emph{Our solution}: 
To provide fully transparent overview of the statistical analyses from raw data
to interpretable results we propose our tool \emph{walrus}. It lets users create
and run analysis pipelines in bioinformatics to e.g. analyze high-throughput
sequencing datasets. In addition, it tracks full provenance of the input,
intermediate, and output data, as well as tool parameters. With \emph{walrus} we
have successfully built analysis pipelines to detect somatic mutations in breast
cancer patients. 

To develop applications that interface with the underlying statistical analyses
we have built \emph{Kvik}. Kvik allows applications written in any modern
programming language to interface with the wealth of bioinformatics packages in
the R programming language, as well as information available through online
databases. We have used Kvik to develop the MIxT system for exploring and
comparing transcriptional profiles from blood and tumor samples in breast cancer
patients. 

\emph{Advantages of our solution}:
Our analysis solution has the benefit that it integrates with modern version
control systems to provide provenance information on datasets. It also runs on a
wide range of software platforms and is targeted to the compute infrastructure
found in hospital environment. 

Our solution to build data explorations provides the benefit of developers being
able to develop applications in any programming language. It also facilitates
the reuse and interface with the wealth of statistical packages directly from a
user-interface rather than analysis scripts. 

\emph{Methods}:
% Litt usikker her LA? 

\emph{Long term contributions}:
Discuss why our approach works out nicely? Benefit for future use? Clinical
apps? 

\emph{Thesis statement}:
% Denne må jeg ha hjelp med! :) 
A holistic approach to data analysis and exploration of high-throughput
dataasets in bioinformatics??? 

% alle stegene: pre-proessering, analyser, vise apps 
% hovedresultat: kapt 2
% exp: bcb-paperet 
% pipppelin epaper

\section{List of papers} 
\begin{itemize}
    \item
        \emph{Kvik: three-tier data exploration tools for flexible analysis of
        genomic data in epidemiological studies}
        \\
        \textbf{Bjørn Fjukstad}, Karina Standahl Olsen, Mie Jareid, Eiliv Lund,
        Lars Ailo Bongo. 
        \\ 
        F1000Research 2015.
        
    \item 
        \emph{Building Applications For Interactive Data Exploration In Systems
        Biology.}
        \\
        \textbf{Bjørn Fjukstad}, Vanessa Dumeaux, Karina Standahl Olsen, Michael
        Hallett, Eiliv Lund, Lars Ailo Bongo.  
        \\ 
        The 8th ACM Conference on Bioinformatics, Computational Biology, and
        Health Informatics (ACM BCB) 2017.

    \item 
        \emph{Interactions between the tumor and the blood systemic response of
        breast cancer patients.}
        \\ 
        Vanessa Dumeaux, \textbf{Bjørn Fjukstad}, Hans E Fjosne, Jan-Ole
        Frantzen, Marit Muri Holmen, Enno Rodegerdts, Ellen Schlichting,
        Anne-Lise Børresen-Dale, Lars Ailo Bongo, Eiliv Lund, Michael Hallett.
        \\ 
        PLoS Computational Biology 2017.

    \item \emph{A Review of Scalable Bioinformatics Pipelines} 
        \\
        \textbf{Bjørn Fjukstad}, Lars Ailo Bongo.
        \\ 
        Data Science and Engineering 2017.

    \item \emph{Reproducible Data Analysis Pipelines in Precision Medicine}
        \\
        \textbf{Bjørn Fjukstad}, Vanessa Dumeaux, Michael Hallett, Lars Ailo
        Bongo
        \\
        Conference TBA 2018. 


    \item \emph{nsroot: Minimalist Process Isolation Tool Implemented With Linux
        Namespaces.}
        \\
        Inge Alexander Raknes, \textbf{Bjørn Fjukstad}, Lars Ailo Bongo 
        \\
        NIK 2017. 
        
\end{itemize} 



