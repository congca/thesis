% observation
There is a rapid growth in the number of available biological datasets due to
the decreaseing cost of data collection.  This brings opportunities to gain
novel insights to the underlying biological mechanisms in the development and
progression of diseases such as cancer.  The wide range of different biological
datasets has led to the development of a wealth of software packages and systems
to explore and analyze these datasets.  However, these tools typically target
only small portions of the deep analysis pipeline required to transform the raw
data into interpretable results. While the tools are used to provide novel
insights in diseases, there is little emphasis on reporting and sharing
information about tool versions, input parameters, and other information that
can help others use the same methods on their own datasets. This leads to
unneccesary difficilties to reuse known methods, and difficulties in reproducing
analyses, which leads to unrealized potential for scientific insights and
commercial use in the available datasets. 

% challenge
There are several challenges for researchers to analyze and explore biological
datasets. These challenges are common for large datasets such as high-throughput
sequencing data that require long-running, deep analysis pipelines, as well as
smaller datasets, such as microarray data, that require complex, but
short-running analysis pipelines.  The first is the time and knowledge required
to find and set up the necessary analysis tools to start analyzing a modern
biological dataset.  The second is ensuring the correct input parameters, tool
versions, database versions, and dataset versions when analyzing, and reporting
analysis results to enable reproducible science. The final challenge is the need
for flexible and highly configurable analysis tools to support the large
variation in study design and data types.

% related 
As a result, there are a wealth of approaches and systems to enable analysis of
the complex biological data. To develop deep analysis pipelines,
Galaxy\cite{galaxy} has for a long time provided a simple interface to set up
and execute analysis pipelines for genomic datasets. However, the Galaxy system
is less effective for explorative and flexible
analyses.\cite{spjuth2015experiences} Pachyderm is a system for
developing more flexible analyses that support comparison of pipeline runs from
different workflow configurations and datasets.\cite{pachyderm} However it has
yet to see wide-spread adpotion in Bioinformatics.  New initiatives such as the
\gls{cwl} provide users a standardized way of describing an analysis pipeline
and has multiple implementations such as the reference implementation
cwl\_runner,\footnote{\url{github.com/common-workflow-language/cwltool}}
Arvados,\cite{arvados} Rabix,\cite{rabix} Toil,\cite{toil} Galaxy,\cite{galaxy}
and AWE.\cite{awe} While these systems provide system for batch-processing of
large datasets, systems such as Shiny and OpenCPU provide interactive interfaces
to the R programming language and the many packages for biological data
processing in Bioconductor.  With the addition of new datasets and methods every
year, it seems that analysis of biological data requires a wide array of
different tools and systems.

% our solution
This dissertation argues that, instead, we can design a unified approach that
integrates disperate tools and data into fully reproducible biological data
analysis frameworks.  In particular, we show how software container
technologies, such as Docker, 
provide the necessary foundation to build reproducible environments for any
analysis pipeline, as well as a suitable environent to package an entire data
expliration application. 

The resulting approach as several key advantages over current systems: 
\begin{itemize} 
    \item
\end{itemize} 

We implement walrus/kvik... 

% We implement the RDD architecture in a stack of open source systems including
% Apache Spark as the common foundation; evaluation: 
From a longer-term perspective, ....

\emph{Thesis statement}:
A hollistic approach based on ...containers ... efficient analsysis of diverse
biological datasets ... 
% hva hollistic can ... support diverse distr. comp.
% A holistic approach to data analysis and exploration of high-throughput
% dataasets in bioinformatics??? 

\section{Problems with Data Analysis and Exploration in Bioinformatics} 
    List of things that we want to fix. 

%  no way I'm keeping the acronym    
\section{The Container-based Data Analysis Model (CDAM)} 
    The Solution (?) 

\section{Use of the CDAM} 

To provide an overview of the statistical analyses from raw data to
interpretable results we propose our tool \emph{walrus}. It lets users create
and run analysis pipelines in bioinformatics to e.g. analyze high-throughput
sequencing datasets. In addition, it tracks full provenance of the input,
intermediate, and output data, as well as tool parameters. With \emph{walrus} we
have successfully built analysis pipelines to detect somatic mutations in breast
cancer patients. 

To develop applications that interface with the underlying statistical analyses
we have built \emph{Kvik}. Kvik allows applications written in any modern
programming language to interface with the wealth of bioinformatics packages in
the R programming language, as well as information available through online
databases. We have used Kvik to develop the MIxT system for exploring and
comparing transcriptional profiles from blood and tumor samples in breast cancer
patients. 


\textbf{walrus}
\textbf{kvik} 
\textbf{mixt} 

\section{Summary of Results} 

\section{List of papers} 
\begin{itemize}
    \item
        \emph{Kvik: three-tier data exploration tools for flexible analysis of
        genomic data in epidemiological studies}
        \\
        \textbf{Bjørn Fjukstad}, Karina Standahl Olsen, Mie Jareid, Eiliv Lund,
        Lars Ailo Bongo. 
        \\ 
        F1000Research 2015.
        
    \item 
        \emph{Building Applications For Interactive Data Exploration In Systems
        Biology.}
        \\
        \textbf{Bjørn Fjukstad}, Vanessa Dumeaux, Karina Standahl Olsen, Michael
        Hallett, Eiliv Lund, Lars Ailo Bongo.  
        \\ 
        The 8th ACM Conference on Bioinformatics, Computational Biology, and
        Health Informatics (ACM BCB) 2017.

    \item 
        \emph{Interactions between the tumor and the blood systemic response of
        breast cancer patients.}
        \\ 
        Vanessa Dumeaux, \textbf{Bjørn Fjukstad}, Hans E Fjosne, Jan-Ole
        Frantzen, Marit Muri Holmen, Enno Rodegerdts, Ellen Schlichting,
        Anne-Lise Børresen-Dale, Lars Ailo Bongo, Eiliv Lund, Michael Hallett.
        \\ 
        PLoS Computational Biology 2017.

    \item \emph{A Review of Scalable Bioinformatics Pipelines} 
        \\
        \textbf{Bjørn Fjukstad}, Lars Ailo Bongo.
        \\ 
        Data Science and Engineering 2017.

        

    \item \emph{nsroot: Minimalist Process Isolation Tool Implemented With Linux
        Namespaces.}
        \\
        Inge Alexander Raknes, \textbf{Bjørn Fjukstad}, Lars Ailo Bongo 
        \\
        NIK 2017. 


    \item \emph{Transcription factor PAX6 as a novel prognostic factor and
        putative tumour suppressor in non-small cell lung cancer} 
        \\
        Yury Kiselev, Sigve Andersen, Charles Johannessen, \textbf{Bjørn
        Fjukstad}, Karina Standahl Olsen, Helge Stenvold, Samer Al-Saad, Tom
        Dønnem, Elin Richardsen, Roy M Bremnes, and Lill-Tove Rasmussen Busund. 
        \\
        Scientific Reports 2018. 

    \item \emph{Reproducible Data Analysis Pipelines in Precision Medicine}
        \\
        \textbf{Bjørn Fjukstad}, Vanessa Dumeaux, Michael Hallett, Lars Ailo
        Bongo
        \\
        Conference TBA 2018. 
        
\end{itemize} 




\section{Dissertation Plan} 



% et kapitelle per paper (kopi tekst) 
% ekstra related work: 
% diskusjon, coclu
