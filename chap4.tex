We have also used the microservice architecture in an application where users
can upload and explore air pollution data from Northern
Norway.\cite{fjukstad2018low} In the project, air:bit, students from upper
secondary schools in Norway collect air quality data from sensor kits that they
have built and programmed. The web application lets the students upload data
from their kits, and provides a graphical interface for them to explore data
from their own, and other participating schools. The system consists of a web
server frontend that retrieves air pollution data from a backend storage system
to build interactive visualizations. It also integrates the data with other
sources such as the Norwegian Institute for Air Research and the The Norwegian
Meteorological Institute. 

